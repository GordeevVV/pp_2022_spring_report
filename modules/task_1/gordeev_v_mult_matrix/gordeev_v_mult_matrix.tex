\documentclass{report}

\usepackage[warn]{mathtext}
\usepackage[T2A]{fontenc}
\usepackage[utf8]{luainputenc}
\usepackage[english, russian]{babel}
\usepackage[pdftex]{hyperref}
\usepackage{tempora}
\usepackage[12pt]{extsizes}
\usepackage{listings}
\usepackage{color}
\usepackage{geometry}
\usepackage{enumitem}
\usepackage{multirow}
\usepackage{graphicx}
\usepackage{indentfirst}
\usepackage{amsmath}

\geometry{a4paper,top=2cm,bottom=2cm,left=2.5cm,right=1.5cm}
\setlength{\parskip}{0.5cm}
\setlist{nolistsep, itemsep=0.3cm,parsep=0pt}

\usepackage{listings}
\lstset{language=C++,
        basicstyle=\footnotesize,
		keywordstyle=\color{blue}\ttfamily,
		stringstyle=\color{red}\ttfamily,
		commentstyle=\color{green}\ttfamily,
		morecomment=[l][\color{red}]{\#},
		tabsize=4,
		breaklines=true,
  		breakatwhitespace=true,
  		title=\lstname,
}

\makeatletter
\renewcommand\@biblabel[1]{#1.\hfil}
\makeatother

\begin{document}

\begin{titlepage}

\begin{center}
Министерство науки и высшего образования Российской Федерации
\end{center}

\begin{center}
Федеральное государственное автономное образовательное учреждение высшего образования \\
Национальный исследовательский Нижегородский государственный университет им. Н.И. Лобачевского
\end{center}

\begin{center}
Институт информационных технологий, математики и механики
\end{center}

\vspace{4em}

\begin{center}
\textbf{\LargeОтчет по лабораторной работе} \\
\end{center}
\begin{center}
\textbf{\Large«Умножение разреженных матриц. Элементы комплексного типа. Формат хранения матрицы – строковый (CRS).»} \\
\end{center}

\vspace{4em}

\newbox{\lbox}
\savebox{\lbox}{\hbox{text}}
\newlength{\maxl}
\setlength{\maxl}{\wd\lbox}
\hfill\parbox{7cm}{
\hspace*{5cm}\hspace*{-5cm}\textbf{Выполнил:} \\ студент группы 381908-1 \\ Гордеев В. В.\\
\\
\hspace*{5cm}\hspace*{-5cm}\textbf{Проверил:}\\ доцент кафедры МОСТ, \\ кандидат технических наук \\ Сысоев А. В.\\
}
\vspace{\fill}

\begin{center} Нижний Новгород \\ 2022 \end{center}

\end{titlepage}

\setcounter{page}{2}

% Содержание
\tableofcontents
\newpage

% Введение
\section*{Введение}
\addcontentsline{toc}{section}{Введение}
\par Понятие разреженной матрицы можно определить многими способами, суть которых состоит в том, что в разреженной матрице "много" нулевых элементов. Обычно говорят, что матрица разрежена, если она содержит O(n) отличных от нуля элементов. В противном случае матрица считается плотной. Типичным случаем разреженности является ограниченность числа ненулевых элементов в одной строке от 1 до k, где k<<n. Задачи линейной алгебры с разреженными матрицами возникают во многих областях, например, при решении дифференциальных уравнений в частных производных, при решении многомерных задач локальной оптимизации
\newpage

% Постановка задачи
\section*{Постановка задачи}
\addcontentsline{toc}{section}{Постановка задачи}
\par В данном лабораторной работе требуется реализовать последовательную версию и параллельные версии алгоритма умножения разреженных матриц, провести вычислительные эксперименты для сравнения времени работы алгоритмов, используя при этом Google Test, продемонстрировать работу алгоритмов проверкой тестов в консоли, сделать выводы об эффективности реализованных алгоритмов, на основе технологий, позволяющих провести анализ времени работы алгоритмов.
\par Параллельные алгоритмы должны быть реализованы при помощи технологий OpenMP, TBB, std::thread.
\newpage

% Описание алгоритма
\section*{Описание алгоритма}
\addcontentsline{toc}{section}{Описание алгоритма}
\par Алгоритм умножения разреженных матриц:
\begin{enumerate}
\item Перемножение соответственных строк и столбцов двух матриц, при условии, что количество строк одной из матриц совпадает с количеством столбцов другой матрицы.
\newpage

% Описание схемы распараллеливания
\section*{Описание схемы распараллеливания}
\addcontentsline{toc}{section}{Описание схемы распараллеливания}
\par Распараллеливание происходит с помощью перемножения каждой строки первой матрицы с каждым из столбцов второй матрицы в каждом из потоков. В OpenMP и TBB версиях распределение данных осуществляется автоматически. В std::thread версии количество данных, обрабатываемое каждым потоком, определяется по количеству строк/столбцов при перемножении. При выходе из параллельного региона во всех параллельных версиях используется автоматическая синхронизация.

\newpage

% Описание программной реализации
\section*{Описание программной реализации}
\addcontentsline{toc}{section}{Описание программной реализации}
Программа состоит из заголовочного файла mult_matrix.h и двух файлов исходного кода mult_matrix и main.cpp.
\par В заголовочном файле находятся прототипы функций алгоритмов последовательного и параллельного перемножения разреженных матриц.
\par Функция для получения пустой матрицы заданного размера:
\begin{lstlisting}
std::vector<std::vector<std::complex<int>>> getEmptyMatrix(int _i, int _j);
\end{lstlisting}
Первый параметр функции является высотой матрицы, второй параметр - длиной матрицы.
\par Функция для получения заполненной разреженной матрицы:
\begin{lstlisting}
std::vector<std::vector<std::complex<int>>> getRandomMatrix(int _i, int _j);
\end{lstlisting}
Первый параметр функции является высотой уже созданной матрицы, второй параметр - ее длиной.
\par Функции для получения длины и ширины матрицы:
\begin{lstlisting}
int getLength() const;
\end{lstlisting}
\begin{lstlisting}
int getHeight() const;
\end{lstlisting}
\par Функция для получения элемента матрицы:
\begin{lstlisting}
std::complex<int> getElement(int row, int col) const;
\end{lstlisting}
Первый параметр это номер строки в матрице, второй - номер столбца.
\par Функция для последовательной версии алгоритма:
\begin{lstlisting}
MultMatrix multMatrixSequential(const MultMatrix& mtx);
\end{lstlisting}
Входящий параметр - ссылка на объект второй для перемножения матрицы.
\par Функция для параллельного алгоритма (OpenMP версия):
\begin{lstlisting}
MultMatrix multMatrixParallel(const MultMatrix& mtx);
\end{lstlisting}
\par Функция для параллельного алгоритма (TBB версия):
\begin{lstlisting}
MultMatrix multMatrixParallel(const MultMatrix& mtx);
\end{lstlisting}
\par Функция для параллельного алгоритма (std::thread версия):
\begin{lstlisting}
MultMatrix multMatrixParallel(const MultMatrix& mtx);
\end{lstlisting}
Входные параметры параллельных функций совпадают с входными параметрами функции для последовательного алгоритма.
\par В файле исходного кода mult_matrix.cpp содержится реализация функций, объявленных в заголовочном файле. В файле исходного кода main.cpp содержатся тесты для проверки корректности программы.
\newpage

% Подтверждение корректности
\section*{Подтверждение корректности}
\addcontentsline{toc}{section}{Подтверждение корректности}
Для подтверждения корректности работы данной программы с помощью фрэймфорка Google Test мной было разработано 6 тестов для каждой версии алгоритма. В каждом из тестов осуществляется проверка части функционала разработанного мной класса для перемножения матриц, в том числе и проверка параллельной и последовательной версий, а так же подсчитывается время работы обоих алгоритмов, находится ускорение и затем результаты, полученные последовательным и параллельным способом сравниваются между собой.
\par Всеми алгоритмами были достигнуты правильные результаты при перемножении матриц.
\par Успешное прохождение всех тестов подтверждает корректность работы программы.
\newpage

% Результаты экспериментов
\section*{Результаты экспериментов}
\addcontentsline{toc}{section}{Результаты экспериментов}
Вычислительные эксперименты для оценки эффективности работы параллельных алгоритмов проводились на ПК со следующими характеристиками:
\begin{itemize}
\item Процессор: AMD Ryzen 5 5600H, 3.30 ГГц, количество ядер: 6;
\item Оперативная память: 16 ГБ (DDR4), 3200 МГц;
\item Операционная система: Windows 10 Home.
\end{itemize}

\par Эксперименты проводились на 6 потоках. Размеры матриц 200*300 и 300*400.

\par Результаты экспериментов.
\begin{table}[!h]
\caption{Результаты вычислительных экспериментов}
\centering
\begin{tabular}{| p{2cm} | p{3cm} | p{4cm} | p{2cm} |}
\hline
Версия алгоритма & Время работы (в секундах) & Ускорение  \\[5pt]
\hline
Sequential        & 1.57611        & -         \\
OpenMP        & 0.369671        & 1.60312          \\
TBB       & 0.262426        & 1.98169         \\
std::thread        & 0.43303        & 1.54777           \\

\hline
\end{tabular}
\end{table}

\newpage

% Выводы из результатов экспериментов
\section*{Выводы из результатов экспериментов}
\addcontentsline{toc}{section}{Выводы из результатов экспериментов}
Можно заметить, что параллельные версии всегда быстрее последовательных примерно 6 раз, это говорит о том, что имеется связь между количеством физических ядер в компьютере и попыткой распараллеливания программы, в omp и tbb версиях используется не совсем чистое распараллеливание, что то остается под капотом, поэтому там различие немного больше, но в std::thread распараллеливание напрямую отражено в программе поэтому там отличие максимально зависит от количества используемых потоков, погрешность в остальных методах, это может быть механизм кэширования, алгоритмы планирования операционной системы и внутренняя реализация.
\newpage

% Заключение
\section*{Заключение}
\addcontentsline{toc}{section}{Заключение}
Было проведено исследование теоретической части алгоритма умножения разреженных матриц, а так же реализованы последовательный и параллельные методы умножения разреженных матриц. Проведенные тесты показали корректность реализованной программы, а проведенные эксперименты доказали эффективность параллельных алгоритмов по сравнению с последовательным.
\newpage

% Литература
\section*{Литература}
\addcontentsline{toc}{section}{Литература}
\begin{enumerate}
\item ИНТУИТ - Электронный ресурс. URL: \newline \url{https://intuit.ru/studies/courses/4447/983/lecture/14931?page=5}
\item Мееров И.Б., Сысоев А.В. «Разреженное матричное умножение». URL: \newline \url{http://www.hpcc.unn.ru/file.php?id=486}
\end{enumerate}
\newpage

% Приложение
\section*{Приложение}
\addcontentsline{toc}{section}{Приложение}
\textbf{Последовательная версия}
\newline
\newline mult_matrix.h
\begin{lstlisting}
// Copyright 2022 Gordeev Viktor
#ifndef MODULES_TASK_1_GORDEEV_V_MULT_COMPLEX_MATRIX_MULT_COMPLEX_MATRIX_H_
#define MODULES_TASK_1_GORDEEV_V_MULT_COMPLEX_MATRIX_MULT_COMPLEX_MATRIX_H_


#include <vector>
#include <string>
#include <stdexcept>
#include <complex>

class MultMatrix {
 private:
    int i;
    int j;
    std::vector<int> rows;
    std::vector<int> columns;
    std::vector<std::complex<int>> mtxVector;

    std::vector<std::vector<std::complex<int>>> getRandomMatrix(int _i, int _j);
    std::vector<std::vector<std::complex<int>>> getEmptyMatrix(int _i, int _j);
 public:
    MultMatrix(int i, int j);
    explicit MultMatrix(std::vector<std::vector<std::complex<int>>> mtx);
    int getLength() const;
    int getHeight() const;
    std::complex<int> getElement(int row, int col) const;
    MultMatrix multMatrix(const MultMatrix &matrix);
};

#endif  // MODULES_TASK_1_GORDEEV_V_MULT_COMPLEX_MATRIX_MULT_COMPLEX_MATRIX_H_

\end{lstlisting}
mult_matrix.cpp
\begin{lstlisting}
// Copyright 2022 Gordeev Viktor
#include <vector>
#include <random>

#include "../../../modules/task_1/gordeev_v_mult_complex_matrix/mult_complex_matrix.h"

std::vector<std::vector<std::complex<int>>>
        MultMatrix::getRandomMatrix(int _i, int _j) {
    std::random_device dev;
    std::mt19937 gen(dev());
    std::vector<std::vector<std::complex<int>>> result;
    for (int k = 0; k < _i; k++) {
        std::vector<std::complex<int>> tmp;
        for (int l = 0; l < _j; l++) {
            if (gen() % 1000 > 350) {
                std::complex<int> complexNumber(0, 0);
                tmp.push_back(complexNumber);
            } else {
                std::complex<int> complexNumber(gen() % 10, (gen() % 10));
                tmp.push_back(complexNumber);
            }
        }
        result.push_back(tmp);
    }
    return result;
}

std::vector<std::vector<std::complex<int>>>
        MultMatrix::getEmptyMatrix(int _i, int _j) {
    std::vector<std::vector<std::complex<int>>> mtx;
    for (int k = 0; k < _i; k++) {
        mtx.push_back(std::vector<std::complex<int>>(_j));
    }
    return mtx;
}

MultMatrix::MultMatrix(int _i, int _j) {
    std::vector<std::vector<std::complex<int>>> mtx = getRandomMatrix(_i, _j);
    i = mtx.size();
    j = mtx[0].size();
    std::complex<int> zero(0, 0);
    rows.push_back(0);
    int counter = 0;
    for (int k = 0; k < static_cast<int>(mtx.size()); k++) {
        for (int l = 0; l < static_cast<int>(mtx[0].size()); l++) {
            if (mtx[k][l] != zero) {
                mtxVector.push_back(mtx[k][l]);
                columns.push_back(l);
                counter++;
            }
        }
        rows.push_back(counter);
    }
}

MultMatrix::MultMatrix(std::vector<std::vector<std::complex<int>>> mtx) {
    i = mtx.size();
    j = mtx[0].size();
    std::complex<int> zero(0, 0);
    rows.push_back(0);
    int counter = 0;
    for (int k = 0; k < static_cast<int>(mtx.size()); k++) {
        for (int l = 0; l < static_cast<int>(mtx[0].size()); l++) {
            if (mtx[k][l] != zero) {
                mtxVector.push_back(mtx[k][l]);
                columns.push_back(l);
                counter++;
            }
        }
        rows.push_back(counter);
    }
}

int MultMatrix::getLength() const {
    return i;
}

int MultMatrix::getHeight() const {
    return j;
}

std::complex<int> MultMatrix::getElement(int row, int column) const {
    if (row > i || column > j || row < 0 || column < 0) {
        throw -1;
    }
    int currentColumn;
    for (int k = rows[row]; k < rows[row + 1]; k++) {
        currentColumn = columns[k];
        if (currentColumn == column) {
            return mtxVector[k];
        }
    }
    return std::complex<int>();
}

MultMatrix MultMatrix::multMatrix(const MultMatrix &mtx) {
    if (j != mtx.i) {
        throw -1;
    }
    std::complex<int> tmp;
    std::vector<std::vector<std::complex<int>>> res = getEmptyMatrix(i, mtx.j);
    for (int k = 0; k < i; k++) {
        for (int l = 0; l < mtx.j; l++) {
            tmp = std::complex<int>();
            for (int m = 0; m < j; m++) {
                tmp += getElement(k, m) *
                        mtx.getElement(m, l);
            }
            res[k][l] = tmp;
        }
    }
    return MultMatrix(res);
}

\end{lstlisting}
main.cpp
\begin{lstlisting}
// Copyright 2022 Gordeev Viktor
#include <gtest/gtest.h>
#include <vector>
#include <omp.h>
#include "../../../modules/task_1/gordeev_v_mult_complex_matrix/mult_complex_matrix.h"

TEST(Gordeev_Mult_Matrix, Constructor_With_No_Throw) {
    EXPECT_NO_THROW(MultMatrix mtx(5, 5));
}

TEST(Gordeev_Mult_Matrix, Cant_Get_Wrong_Matrix_Element) {
    MultMatrix mtx(1, 3);
    EXPECT_ANY_THROW(mtx.getElement(4, 4));
}

TEST(Gordeev_Mult_Matrix, Can_Get_Matrix_Element) {
    MultMatrix mtx(10, 10);
    EXPECT_NO_THROW(mtx.getElement(4, 4));
}

TEST(Gordeev_Mult_Matrix, Can_Mult_Matrix_Works_Correct) {
    MultMatrix mtx1(3, 5);
    MultMatrix mtx2(5, 6);
    EXPECT_NO_THROW(mtx1.multMatrix(mtx2));
}

TEST(Gordeev_Mult_Matrix, Cant_Mult_Wrong_Size_Matrix) {
    MultMatrix mtx1(2, 9);
    MultMatrix mtx2(4, 5);
    EXPECT_ANY_THROW(mtx1.multMatrix(mtx2));
}

TEST(Gordeev_Mult_Matrix, Res_Mult_Correct_Size) {
    MultMatrix mtx1(4, 10);
    MultMatrix mtx2(10, 10);
    MultMatrix res = mtx1.multMatrix(mtx2);
    EXPECT_EQ(res.getLength(), mtx1.getLength());
    EXPECT_EQ(res.getHeight(), mtx2.getHeight());
}
TEST(Gordeev_Mult_Matrix, Time_Test) {
    MultMatrix mtx1(200, 300);
    MultMatrix mtx2(300, 400);
    volatile double start_seq = omp_get_wtime();
    MultMatrix res1 = mtx1.multMatrix(mtx2);
    volatile double end_seq = omp_get_wtime();

    std::cout<<"seq: "<<start_seq-end_seq<<"\n";
    ASSERT_NO_THROW();
}

int main(int argc, char **argv) {
    ::testing::InitGoogleTest(&argc, argv);
    return RUN_ALL_TESTS();
}


\end{lstlisting}

\textbf{OpenMP версия}
\newline
\newline mul_matrix.h
\begin{lstlisting}
// Copyright 2022 Gordeev Viktor
#ifndef MODULES_TASK_2_GORDEEV_V_MULT_COMPLEX_MATRIX_MULT_COMPLEX_MATRIX_H_
#define MODULES_TASK_2_GORDEEV_V_MULT_COMPLEX_MATRIX_MULT_COMPLEX_MATRIX_H_

#include <omp.h>
#include <vector>
#include <string>
#include <random>
#include <complex>
class MultMatrix {
 private:
    int i;
    int j;
    std::vector<int> rows;
    std::vector<int> columns;
    std::vector<std::complex<int>> mtxVector;

    std::vector<std::vector<std::complex<int>>> getRandomMatrix(int _i, int _j);
    std::vector<std::vector<std::complex<int>>> getEmptyMatrix(int _i, int _j);
 public:
    MultMatrix(int i, int j);
    explicit MultMatrix(std::vector<std::vector<std::complex<int>>> mtx);
    int getLength() const;
    int getHeight() const;
    std::complex<int> getElement(int row, int col) const;
    MultMatrix multMatrixSequential(const MultMatrix& mtx);
    MultMatrix multMatrixParallel(const MultMatrix& mtx);
    bool equal(const MultMatrix& mtx);
};
#endif  // MODULES_TASK_2_GORDEEV_V_MULT_COMPLEX_MATRIX_MULT_COMPLEX_MATRIX_H_

\end{lstlisting}
mult_matrix.cpp
\begin{lstlisting}
// Copyright 2022 Gordeev Viktor

#include "../../../modules/task_2/gordeev_v_mult_complex_matrix/mult_complex_matrix.h"

std::vector<std::vector<std::complex<int>>>
MultMatrix::getRandomMatrix(int _i, int _j) {
    std::random_device dev;
    std::mt19937 gen(dev());
    std::vector<std::vector<std::complex<int>>> result;
    for (int k = 0; k < _i; k++) {
        std::vector<std::complex<int>> tmp;
        for (int l = 0; l < _j; l++) {
            if (gen() % 1000 > 350) {
                std::complex<int> complexNumber(0, 0);
                tmp.push_back(complexNumber);
            } else {
                std::complex<int> complexNumber(gen() % 10, (gen() % 10));
                tmp.push_back(complexNumber);
            }
        }
        result.push_back(tmp);
    }
    return result;
}

std::vector<std::vector<std::complex<int>>>
MultMatrix::getEmptyMatrix(int _i, int _j) {
    std::vector<std::vector<std::complex<int>>> mtx;
    for (int k = 0; k < _i; k++) {
        mtx.push_back(std::vector<std::complex<int>>(_j));
    }
    return mtx;
}

MultMatrix::MultMatrix(int _i, int _j) {
    std::vector<std::vector<std::complex<int>>> mtx = getRandomMatrix(_i, _j);
    i = mtx.size();
    j = mtx[0].size();
    std::complex<int> zero(0, 0);
    rows.push_back(0);
    int counter = 0;
    for (int k = 0; k < static_cast<int>(mtx.size()); k++) {
        for (int l = 0; l < static_cast<int>(mtx[0].size()); l++) {
            if (mtx[k][l] != zero) {
                mtxVector.push_back(mtx[k][l]);
                columns.push_back(l);
                counter++;
            }
        }
        rows.push_back(counter);
    }
}

MultMatrix::MultMatrix(std::vector<std::vector<std::complex<int>>> mtx) {
    i = mtx.size();
    j = mtx[0].size();
    std::complex<int> zero(0, 0);
    rows.push_back(0);
    int counter = 0;
    for (int k = 0; k < static_cast<int>(mtx.size()); k++) {
        for (int l = 0; l < static_cast<int>(mtx[0].size()); l++) {
            if (mtx[k][l] != zero) {
                mtxVector.push_back(mtx[k][l]);
                columns.push_back(l);
                counter++;
            }
        }
        rows.push_back(counter);
    }
}

int MultMatrix::getLength() const {
    return i;
}

int MultMatrix::getHeight() const {
    return j;
}

std::complex<int> MultMatrix::getElement(int row, int column) const {
    if (row > i || column > j || row < 0 || column < 0) {
        throw -1;
    }
    int currentColumn;
    for (int k = rows[row]; k < rows[row + 1]; k++) {
        currentColumn = columns[k];
        if (currentColumn == column) {
            return mtxVector[k];
        }
    }
    return std::complex<int>();
}

MultMatrix MultMatrix::multMatrixSequential(const MultMatrix &mtx) {
    if (j != mtx.i) {
        throw -1;
    }
    std::complex<int> tmp;
    std::vector<std::vector<std::complex<int>>> res = getEmptyMatrix(i, mtx.j);
    for (int k = 0; k < i; k++) {
        for (int l = 0; l < mtx.j; l++) {
            tmp = std::complex<int>();
            for (int m = 0; m < j; m++) {
                tmp += getElement(k, m) *
                       mtx.getElement(m, l);
            }
            res[k][l] = tmp;
        }
    }
    return MultMatrix(res);
}

MultMatrix MultMatrix::multMatrixParallel(const MultMatrix &mtx) {
    if (j != mtx.i) {
        throw -1;
    }
    std::complex<int> tmp;
    std::vector<std::vector<std::complex<int>>> res = getEmptyMatrix(i, mtx.j);
#pragma omp parallel for shared(mtx, res) private(tmp)
    for (int k = 0; k < i; k++) {
        for (int l = 0; l < mtx.j; l++) {
            tmp = std::complex<int>();
            for (int m = 0; m < j; m++) {
                tmp += getElement(k, m) *
                       mtx.getElement(m, l);
            }
            res[k][l] = tmp;
        }
    }
    return MultMatrix(res);
}

bool MultMatrix::equal(const MultMatrix &mtx) {
    if (i != mtx.i)
        return false;
    if (j != mtx.j)
        return false;
    if (mtxVector != mtx.mtxVector)
        return false;
    return true;
}

\end{lstlisting}
main.cpp
\begin{lstlisting}
// Copyright 2022 Gordeev Viktor

#include <gtest/gtest.h>
#include "../../../modules/task_2/gordeev_v_mult_complex_matrix/mult_complex_matrix.h"

TEST(Gordeev_Mult_Matrix, Constructor_With_No_Throw) {
    EXPECT_NO_THROW(MultMatrix mtx(5, 5));
}

TEST(Gordeev_Mult_Matrix, Cant_Get_Wrong_Matrix_Element) {
    MultMatrix mtx(1, 3);
    EXPECT_ANY_THROW(mtx.getElement(4, 4));
}

TEST(Gordeev_Mult_Matrix, Can_Get_Matrix_Element) {
    MultMatrix mtx(10, 10);
    EXPECT_NO_THROW(mtx.getElement(4, 4));
}

TEST(Gordeev_Mult_Matrix, Can_Mult_Matrix_Works_Correct) {
    MultMatrix mtx1(3, 5);
    MultMatrix mtx2(5, 6);
    EXPECT_NO_THROW(mtx1.multMatrixSequential(mtx2));
}

TEST(Gordeev_Mult_Matrix, Cant_Mult_Wrong_Size_Matrix) {
    MultMatrix mtx1(2, 9);
    MultMatrix mtx2(4, 5);
    EXPECT_ANY_THROW(mtx1.multMatrixSequential(mtx2));
}

TEST(Gordeev_Mult_Matrix, Res_Mult_Correct_Size) {
    MultMatrix mtx1(4, 10);
    MultMatrix mtx2(10, 10);
    MultMatrix res = mtx1.multMatrixSequential(mtx2);
    EXPECT_EQ(res.getLength(), mtx1.getLength());
    EXPECT_EQ(res.getHeight(), mtx2.getHeight());
}

TEST(Gordeev_Omp_Mult_Matrix, Res_Parallel_Equal_To_Sequential) {
    MultMatrix mtx1(2, 3);
    MultMatrix mtx2(3, 4);
    MultMatrix res1 = mtx1.multMatrixSequential(mtx2);
    MultMatrix res2 = mtx1.multMatrixParallel(mtx2);
    ASSERT_TRUE(res1.equal(res2));
}

TEST(GORDEEV_OMP_MULT_MATRIX, Effectivness_Test){
    MultMatrix mtx1(200, 300);
    MultMatrix mtx2(300, 400);

    volatile double start_seq = omp_get_wtime();
    MultMatrix res1 = mtx1.multMatrixSequential(mtx2);
    volatile double end_seq = omp_get_wtime();

    volatile double start_par = omp_get_wtime();
    MultMatrix res2 = mtx1.multMatrixParallel(mtx2);
    volatile double end_par = omp_get_wtime();

    double seq_time = end_seq - start_seq;
    double par_time = end_par - start_par;
    std::cout<<"parallel: "<<start_par-end_par<<"\n";
    std::cout<<"seq: "<<start_seq-end_seq<<"\n";

    ASSERT_NO_THROW();
}

int main(int argc, char **argv) {
    ::testing::InitGoogleTest(&argc, argv);
    return RUN_ALL_TESTS();
}


\end{lstlisting}

\textbf{TBB версия}
\newline
\newline mult_matrix.h
\begin{lstlisting}
// Copyright 2022 Gordeev Viktor
#ifndef MODULES_TASK_3_GORDEEV_V_MULT_COMPLEX_MATRIX_MULT_COMPLEX_MATRIX_H_
#define MODULES_TASK_3_GORDEEV_V_MULT_COMPLEX_MATRIX_MULT_COMPLEX_MATRIX_H_

#include <tbb/tbb.h>
#include <vector>
#include <string>
#include <random>
#include <iostream>
#include <complex>


class MultMatrix {
 private:
    int i;
    int j;
    std::vector<int> rows;
    std::vector<int> columns;
    std::vector<std::complex<int>> mtxVector;

    std::vector<std::vector<std::complex<int>>> getRandomMatrix(int _i, int _j);
    std::vector<std::vector<std::complex<int>>> getEmptyMatrix(int _i, int _j);
 public:
    MultMatrix(int i, int j);
    explicit MultMatrix(std::vector<std::vector<std::complex<int>>> mtx);
    int getLength() const;
    int getHeight() const;
    std::complex<int> getElement(int row, int col) const;
    MultMatrix multMatrixSequential(const MultMatrix& mtx);
    MultMatrix multMatrixParallel(const MultMatrix& mtx);
    bool equal(const MultMatrix& mtx);
};

#endif  // MODULES_TASK_3_GORDEEV_V_MULT_COMPLEX_MATRIX_MULT_COMPLEX_MATRIX_H_

\end{lstlisting}
mult_matrix.cpp
\begin{lstlisting}
// Copyright 2022 Gordeev Viktor

#include "../../../modules/task_3/gordeev_v_mult_complex_matrix/mult_complex_matrix.h"

std::vector<std::vector<std::complex<int>>>

MultMatrix::getRandomMatrix(int _i, int _j) {
    std::random_device dev;
    std::mt19937 gen(dev());
    std::vector<std::vector<std::complex<int>>> result;
    for (int k = 0; k < _i; k++) {
        std::vector<std::complex<int>> tmp;
        for (int l = 0; l < _j; l++) {
            if (gen() % 1000 > 350) {
                std::complex<int> complexNumber(0, 0);
                tmp.push_back(complexNumber);
            } else {
                std::complex<int> complexNumber(gen() % 10, (gen() % 10));
                tmp.push_back(complexNumber);
            }
        }
        result.push_back(tmp);
    }
    return result;
}

std::vector<std::vector<std::complex<int>>>

MultMatrix::getEmptyMatrix(int _i, int _j) {
    std::vector<std::vector<std::complex<int>>> mtx;
    for (int k = 0; k < _i; k++) {
        mtx.push_back(std::vector<std::complex<int >>(_j));
    }
    return mtx;
}

MultMatrix::MultMatrix(int _i, int _j) {
    std::vector<std::vector<std::complex<int>>> mtx = getRandomMatrix(_i, _j);
    i = mtx.size();
    j = mtx[0].size();
    std::complex<int> zero(0, 0);
    rows.push_back(0);
    int counter = 0;
    for (int k = 0; k < static_cast<int>(mtx.size()); k++) {
        for (int l = 0; l < static_cast<int>(mtx[0].size()); l++) {
            if (mtx[k][l] != zero) {
                mtxVector.push_back(mtx[k][l]);
                columns.push_back(l);
                counter++;
            }
        }
        rows.push_back(counter);
    }
}

MultMatrix::MultMatrix(std::vector<std::vector<std::complex<int>>> mtx) {
    i = mtx.size();
    j = mtx[0].size();
    std::complex<int> zero(0, 0);
    rows.push_back(0);
    int counter = 0;
    for (int k = 0; k < static_cast<int>(mtx.size()); k++) {
        for (int l = 0; l < static_cast<int>(mtx[0].size()); l++) {
            if (mtx[k][l] != zero) {
                mtxVector.push_back(mtx[k][l]);
                columns.push_back(l);
                counter++;
            }
        }
        rows.push_back(counter);
    }
}

int MultMatrix::getLength() const {
    return i;
}

int MultMatrix::getHeight() const {
    return j;
}

std::complex<int> MultMatrix::getElement(int row, int column) const {
    if (row > i || column > j || row < 0 || column < 0) {
        throw -1;
    }
    int currentColumn;
    for (int k = rows[row]; k < rows[row + 1]; k++) {
        currentColumn = columns[k];
        if (currentColumn == column) {
            return mtxVector[k];
        }
    }
    return std::complex<int>();
}

MultMatrix MultMatrix::multMatrixSequential(const MultMatrix &mtx) {
    if (j != mtx.i) {
        throw -1;
    }
    std::complex<int> tmp;
    std::vector<std::vector<std::complex<int>>> res = getEmptyMatrix(i, mtx.j);
    for (int k = 0; k < i; k++) {
        for (int l = 0; l < mtx.j; l++) {
            tmp = std::complex<int>();
            for (int m = 0; m < j; m++) {
                tmp += getElement(k, m) *
                       mtx.getElement(m, l);
            }
            res[k][l] = tmp;
        }
    }
    return MultMatrix(res);
}

MultMatrix MultMatrix::multMatrixParallel(const MultMatrix &mtx) {
    if (j != mtx.i) {
        throw -1;
    }
    std::vector<std::vector<std::complex<int>>> res = getEmptyMatrix(i, mtx.j);
    tbb::parallel_for(tbb::blocked_range<int>(0, i),
                      [this, &mtx, &res](tbb::blocked_range<int> &z) {
                          std::complex<int> tmp;
                          for (int k = z.begin(); k != z.end(); k++) {
                              for (int l = 0; l < mtx.j; l++) {
                                  tmp = std::complex<int>();
                                  for (int m = 0; m < j; m++) {
                                      tmp += getElement(k, m)
                                             * mtx.getElement(m, l);
                                  }
                                  res[k][l] = tmp;
                              }
                          }
                      });
    return MultMatrix(res);
}

bool MultMatrix::equal(const MultMatrix &mtx) {
    if (i != mtx.i)
        return false;
    if (j != mtx.j)
        return false;
    if (mtxVector != mtx.mtxVector)
        return false;
    return true;
}

\end{lstlisting}
main.cpp
\begin{lstlisting}
// Copyright 2022 Gordeev Viktor

#include <gtest/gtest.h>

#include "../../../modules/task_3/gordeev_v_mult_complex_matrix/mult_complex_matrix.h"

TEST(Gordeev_Mult_Matrix, Constructor_With_No_Throw) {
    EXPECT_NO_THROW(MultMatrix mtx(5, 5));
}

TEST(Gordeev_Mult_Matrix, Cant_Get_Wrong_Matrix_Element) {
    MultMatrix mtx(1, 3);
    EXPECT_ANY_THROW(mtx.getElement(4, 4));
}

TEST(Gordeev_Mult_Matrix, Can_Get_Matrix_Element) {
    MultMatrix mtx(10, 10);
    EXPECT_NO_THROW(mtx.getElement(4, 4));
}

TEST(Gordeev_Mult_Matrix, Can_Mult_Matrix_Works_Correct) {
    MultMatrix mtx1(3, 5);
    MultMatrix mtx2(5, 6);
    EXPECT_NO_THROW(mtx1.multMatrixSequential(mtx2));
}

TEST(Gordeev_Mult_Matrix, Cant_Mult_Wrong_Size_Matrix) {
    MultMatrix mtx1(2, 9);
    MultMatrix mtx2(4, 5);
    EXPECT_ANY_THROW(mtx1.multMatrixSequential(mtx2));
}

TEST(Gordeev_Mult_Matrix, Res_Mult_Correct_Size) {
    MultMatrix mtx1(4, 10);
    MultMatrix mtx2(10, 10);
    MultMatrix res = mtx1.multMatrixSequential(mtx2);
    EXPECT_EQ(res.getLength(), mtx1.getLength());
    EXPECT_EQ(res.getHeight(), mtx2.getHeight());
}

TEST(Gordeev_Tbb_Mult_Matrix, Res_Parallel_Equal_To_Sequential) {
    MultMatrix mtx1(2, 3);
    MultMatrix mtx2(3, 4);
    MultMatrix res1 = mtx1.multMatrixSequential(mtx2);
    MultMatrix res2 = mtx1.multMatrixParallel(mtx2);
    ASSERT_TRUE(res1.equal(res2));
}

 TEST(Gordeev_Tbb_Mult_Matrix, Test_Time) {
    MultMatrix mtx1(200, 300);
    MultMatrix mtx2(300, 400);

    tbb::tick_count startSequential = tbb::tick_count::now();
    MultMatrix res1 = mtx1.multMatrixSequential(mtx2);
    tbb::tick_count endSequential = tbb::tick_count::now();

    tbb::tick_count startParallel = tbb::tick_count::now();
    MultMatrix res2 = mtx1.multMatrixParallel(mtx2);
    tbb::tick_count endParallel = tbb::tick_count::now();

    double seq_time = (endSequential - startSequential).seconds();
    double par_time = (endParallel - startParallel).seconds();
    std::cout<<"parallel: "<<par_time<<"\n";
    std::cout<<"seq: "<<seq_time<<"\n";
    std::cout << "Effectiveness: " << seq_time / par_time << '\n';
    ASSERT_NO_THROW();
 }


int main(int argc, char **argv) {
    ::testing::InitGoogleTest(&argc, argv);
    return RUN_ALL_TESTS();
}

\end{lstlisting}

\textbf{std::thread версия}
\newline
\newline mult_matrix.h
\begin{lstlisting}
// Copyright 2022 Gordeev Viktor

#ifndef MODULES_TASK_4_GORDEEV_V_MULT_COMPLEX_MATRIX_MULT_COMPLEX_MATRIX_H_
#define MODULES_TASK_4_GORDEEV_V_MULT_COMPLEX_MATRIX_MULT_COMPLEX_MATRIX_H_

#include <complex>
#include <vector>
#include <string>
#include <random>
#include <iostream>
#include <thread> // NOLINT [build/c++11]

class MultMatrix {
 private:
    int i;
    int j;
    std::vector<int> rows;
    std::vector<int> columns;
    std::vector<std::complex<int>> mtxVector;

    std::vector<std::vector<std::complex<int>>> getRandomMatrix(int _i, int _j);
    std::vector<std::vector<std::complex<int>>> getEmptyMatrix(int _i, int _j);
 public:
    MultMatrix(int i, int j);
    explicit MultMatrix(std::vector<std::vector<std::complex<int>>> mtx);
    int getLength() const;
    int getHeight() const;
    std::complex<int> getElement(int row, int col) const;
    MultMatrix multMatrixSequential(const MultMatrix& mtx);
    MultMatrix multMatrixParallel(const MultMatrix& mtx);
    bool equal(const MultMatrix& mtx);
};

#endif  // MODULES_TASK_4_GORDEEV_V_MULT_COMPLEX_MATRIX_MULT_COMPLEX_MATRIX_H_

\end{lstlisting}
mult_matrix.cpp
\begin{lstlisting}
// Copyright 2022 Gordeev Viktor

#include "../../../modules/task_4/gordeev_v_mult_complex_matrix/mult_complex_matrix.h"

std::vector<std::vector<std::complex<int>>>

MultMatrix::getRandomMatrix(int _i, int _j) {
    std::random_device dev;
    std::mt19937 gen(dev());
    std::vector<std::vector<std::complex<int>>> result;
    for (int k = 0; k < _i; k++) {
        std::vector<std::complex<int>> tmp;
        for (int l = 0; l < _j; l++) {
            if (gen() % 1000 > 350) {
                std::complex<int> complexNumber(0, 0);
                tmp.push_back(complexNumber);
            } else {
                std::complex<int> complexNumber(gen() % 10, (gen() % 10));
                tmp.push_back(complexNumber);
            }
        }
        result.push_back(tmp);
    }
    return result;
}

std::vector<std::vector<std::complex<int>>>

MultMatrix::getEmptyMatrix(int _i, int _j) {
    std::vector<std::vector<std::complex<int>>> mtx;
    for (int k = 0; k < _i; k++) {
        mtx.push_back(std::vector<std::complex<int >>(_j));
    }
    return mtx;
}

MultMatrix::MultMatrix(int _i, int _j) {
    std::vector<std::vector<std::complex<int>>> mtx = getRandomMatrix(_i, _j);
    i = mtx.size();
    j = mtx[0].size();
    std::complex<int> zero(0, 0);
    rows.push_back(0);
    int counter = 0;
    for (int k = 0; k < static_cast<int>(mtx.size()); k++) {
        for (int l = 0; l < static_cast<int>(mtx[0].size()); l++) {
            if (mtx[k][l] != zero) {
                mtxVector.push_back(mtx[k][l]);
                columns.push_back(l);
                counter++;
            }
        }
        rows.push_back(counter);
    }
}

MultMatrix::MultMatrix(std::vector<std::vector<std::complex<int>>> mtx) {
    i = mtx.size();
    j = mtx[0].size();
    std::complex<int> zero(0, 0);
    rows.push_back(0);
    int counter = 0;
    for (int k = 0; k < static_cast<int>(mtx.size()); k++) {
        for (int l = 0; l < static_cast<int>(mtx[0].size()); l++) {
            if (mtx[k][l] != zero) {
                mtxVector.push_back(mtx[k][l]);
                columns.push_back(l);
                counter++;
            }
        }
        rows.push_back(counter);
    }
}

int MultMatrix::getLength() const {
    return i;
}

int MultMatrix::getHeight() const {
    return j;
}

std::complex<int> MultMatrix::getElement(int row, int column) const {
    if (row > i || column > j || row < 0 || column < 0) {
        throw -1;
    }
    int currentColumn;
    for (int k = rows[row]; k < rows[row + 1]; k++) {
        currentColumn = columns[k];
        if (currentColumn == column) {
            return mtxVector[k];
        }
    }
    return std::complex<int>();
}

MultMatrix MultMatrix::multMatrixSequential(const MultMatrix &mtx) {
    if (j != mtx.i) {
        throw -1;
    }
    std::complex<int> tmp;
    std::vector<std::vector<std::complex<int>>> res = getEmptyMatrix(i, mtx.j);
    for (int k = 0; k < i; k++) {
        for (int l = 0; l < mtx.j; l++) {
            tmp = std::complex<int>();
            for (int m = 0; m < j; m++) {
                tmp += getElement(k, m) *
                       mtx.getElement(m, l);
            }
            res[k][l] = tmp;
        }
    }
    return MultMatrix(res);
}

MultMatrix MultMatrix::multMatrixParallel(const MultMatrix &mtx) {
    if (j != mtx.i) {
        throw -1;
    }
    std::vector<std::thread> threads;
    std::vector<std::vector<std::complex<int>>> res = getEmptyMatrix(i, mtx.j);
    int threadNumber = std::thread::hardware_concurrency();
    int remains = i % threadNumber;
    int countAll = i / threadNumber;
    for (int k = 0, currentThread = 0; k < i; k+=countAll, currentThread++) {
        int from = k;
        int to = k + countAll;
        if (currentThread == threadNumber - 1 && remains != 0) {
            to += remains;
            k = i;
        }
        threads.push_back(std::thread(
                [from, to, this, &mtx, &res]() {
                    std::complex<int> tmp;
                    for (int k = from; k < to; k++) {
                        for (int l = 0; l < mtx.j; l++) {
                            tmp = std::complex<int>();
                            for (int m = 0; m < j; m++) {
                                tmp += getElement(k, m)
                                        * mtx.getElement(m, l);
                            }
                            res[k][l] = tmp;
                        }
                    }
                }));
    }
    for (int k = 0; k < static_cast<int>(threads.size()); k++) {
        threads[k].join();
    }
    return MultMatrix(res);
}

bool MultMatrix::equal(const MultMatrix &mtx) {
    if (i != mtx.i)
        return false;
    if (j != mtx.j)
        return false;
    if (mtxVector != mtx.mtxVector)
        return false;
    return true;
}

\end{lstlisting}
main.cpp
\begin{lstlisting}
// Copyright 2022 Gordeev Viktor

#include <gtest/gtest.h>
#include "../../../modules/task_4/gordeev_v_mult_complex_matrix/mult_complex_matrix.h"
#include <omp.h>

TEST(Gordeev_Mult_Matrix, Constructor_With_No_Throw) {
    EXPECT_NO_THROW(MultMatrix mtx(5, 5));
}

TEST(Gordeev_Mult_Matrix, Cant_Get_Wrong_Matrix_Element) {
    MultMatrix mtx(1, 3);
    EXPECT_ANY_THROW(mtx.getElement(4, 4));
}

TEST(Gordeev_Mult_Matrix, Can_Get_Matrix_Element) {
    MultMatrix mtx(10, 10);
    EXPECT_NO_THROW(mtx.getElement(4, 4));
}

TEST(Gordeev_Mult_Matrix, Can_Mult_Matrix_Works_Correct) {
    MultMatrix mtx1(3, 5);
    MultMatrix mtx2(5, 6);
    EXPECT_NO_THROW(mtx1.multMatrixSequential(mtx2));
}

TEST(Gordeev_Mult_Matrix, Cant_Mult_Wrong_Size_Matrix) {
    MultMatrix mtx1(2, 9);
    MultMatrix mtx2(4, 5);
    EXPECT_ANY_THROW(mtx1.multMatrixSequential(mtx2));
}

TEST(Gordeev_Mult_Matrix, Res_Mult_Correct_Size) {
    MultMatrix mtx1(4, 10);
    MultMatrix mtx2(10, 10);
    MultMatrix res = mtx1.multMatrixSequential(mtx2);
    EXPECT_EQ(res.getLength(), mtx1.getLength());
    EXPECT_EQ(res.getHeight(), mtx2.getHeight());
}

TEST(Gordeev_Thread_Mult_Matrix, Res_Parallel_Equal_To_Sequential) {
    MultMatrix mtx1(4, 6);
    MultMatrix mtx2(6, 8);
    MultMatrix res1 = mtx1.multMatrixSequential(mtx2);
    MultMatrix res2 = mtx1.multMatrixParallel(mtx2);
    ASSERT_TRUE(res1.equal(res2));
}

 TEST(Gordeev_Thread_Mult_Matrix, Time_Test){
    MultMatrix mtx1(200, 300);
    MultMatrix mtx2(300, 400);

    volatile double start_seq = omp_get_wtime();
    MultMatrix res1 =mtx1.multMatrixSequential(mtx2);
    volatile double end_seq = omp_get_wtime();

    volatile double start_par = omp_get_wtime();
    MultMatrix res2 = mtx1.multMatrixParallel(mtx2);
    volatile double end_par = omp_get_wtime();
    double seq_time = end_seq - start_seq;
    double par_time = end_par - start_par;
    std::cout<<"parallel: " <<  par_time << "\n";
    std::cout<<"seq: " << seq_time <<"\n";
    std::cout << "Effectiveness: " <<
    seq_time/ par_time << '\n';
    ASSERT_NO_THROW();
 }

int main(int argc, char **argv) {
    ::testing::InitGoogleTest(&argc, argv);
    return RUN_ALL_TESTS();
}

\end{lstlisting}

\end{document}
